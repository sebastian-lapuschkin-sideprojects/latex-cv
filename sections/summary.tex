\roottitle{Short Bio} % Root section title

\vspace{-1.3em} % Reduce whitespace after the Summary heading and the two-column content

\begin{multicols}{2}  % Start a two-column layout
\noindent

Sebastian Lapuschkin is the Head of the Explainable Artificial Intelligence research group at Fraunhofer Heinrich Hertz Institute (HHI) in Berlin.

He received his Ph.D. degree with distinction from the Technische Universität Berlin in 2018
for his pioneering contributions to the field of Explainable Artificial Intelligence (XAI) and interpretable machine learning.
From 2007 to 2013 he studied computer science (B. Sc. and M. Sc.) at the Technische Universität Berlin,
with a focus on software engineering and machine learning.

Sebastian is the recipient of multiple awards, including the Hugo-Geiger-Prize for outstanding doctoral achievement and the 2020 Pattern Recognition Best Paper Award.

%The research of Sebastian
His research has shaped the field of XAI from the very beginning,
with contributions to the first wave of XAI such as the popular and widely-used Layer-wise Relevance Propagation method,
as well as timely works influencing the second wave of XAI with additions to the sub-fields of Mechanistic Interpretability, Data Attribution and XAI-based model- and data improvement.

Sebastian is an avid advocate for Open Science, demonstrated by numerous Free Open Source Software toolboxes published with the intent to warrant and facilitate reproducibility in AI research.

% focused on pushing the boundaries of XAI, e.g, for achieving human-understandable explanations,
% and towards the effective and efficient utilization of interpretable feedback for the improvement of machine learning systems and data.

Since 2024 he is co-organizing The World Conference on eXplainable Artificial Intelligence and serves as a Topic Editor on ``Opportunities and Challenges in Explainable Artificial Intelligence'' for the MDPI Open Access Journals.

Further research interests include efficient machine learning and data analysis, as well as data and algorithm visualization.

\end{multicols}
